\documentclass[a4paper,11pt]{article}
\usepackage[utf8]{inputenc}
\usepackage[a4paper, total={5.9in, 8.75in}]{geometry}
\usepackage{tikz}
\usetikzlibrary{arrows}
\usepackage{subeqnarray}
\usepackage{IEEEtrantools}
\usepackage{listings}
\usepackage{amsfonts}
\usepackage{float}
\usepackage{url}
\usepackage{indentfirst}
\usepackage{mathtools}
%\setlength{\parindent}{0pt}
\usepackage[colorlinks]{hyperref}
\hypersetup{
    colorlinks=true,
    linkcolor=blue,
    filecolor=magenta,      
    urlcolor=cyan,
    citecolor=black
}
\usepackage{amsmath,amssymb,amsthm}
\renewcommand{\qedsymbol}{$\blacksquare$}
\newcommand{\qedwhite}{\hfill \ensuremath{\Box}}
\usepackage{subfigure}
\usepackage{multicol}
\usepackage{graphicx}
\usepackage{physics}
\usepackage{sidecap}

\newcommand{\dg}[1]{\ensuremath{#1^{\dagger}}}

\newcommand{\td}[1]{\ensuremath{\tilde{#1}}}

\newcommand{\id}{\mathbb{1}}

\newcommand{\ch}{\mathcal{E}}

\newcommand{\lv}{\mathcal{L}}

\newcommand{\prob}[1]{\text{Pr}\left(#1 \right)}

\theoremstyle{definition}
\newtheorem{definition}{Definition}

\theoremstyle{remark}
\newtheorem*{notation}{Notation}

\theoremstyle{remark}
\newtheorem{remark}{Remark}

\theoremstyle{plain}
\newtheorem{theorem}{Theorem}

\theoremstyle{plain}
\newtheorem{proposition}{Proposition}

\theoremstyle{plain}
\newtheorem{corollary}{Corollary}

\theoremstyle{definition}
\newtheorem{example}{Example}

\theoremstyle{plain}
\newtheorem{lemma}{Lemma}